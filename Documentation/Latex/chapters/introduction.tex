\section{Introduction}
\textbf{\textit{UniSup}}  is an instantaneous chat application that allows users to exchange short text messages among them.

\subsection{Description}

\textit{UniSup} name is composed by \textit{Uni} that stands for University, which is the main application scope; and \textit{Sup} which is a popular slang abbreviation that stands for \textit{"What’s up?"}.
Every time a \textbf{user} logs in correctly (an authentication check is performed), he/she will be able to see his/her chat history. 
After a click on a specific chat, he/she can visualize the list of the last messages exchanged with that particular contact. Filling the text field and clicking on the \textbf{SEND} button will send a message to the selected contact.
At any time, he/she can start a new conversation with a new contact: it only requires a click on the corresponding button, typing destination username and the text Payload and click on the \textbf{SEND}  button.

When a user logs into the system, he/she will receive every message sent to him/her while he/she was offline. On the contrary, while he/she is online, he/she receives messages on \textbf{REAL TIME} and the interface is automatically updated reporting the new message. Of course, messages within a chat are always displayed in chronological send order, and they are forwarded according to a \textbf{FIFO}  policy.

At the application start, the user will visualize an authentication form: he/she can login with an existing account or register a new one, of course no duplicated usernames are allowed.

From the application Scene, by clicking on the \textbf{LOGOUT}  button, the user logs out the system and goes back to the authentication form. The user can now login again, even with a different account.
\medskip \\
